\documentclass[a4paper,12pt]{article}
\usepackage[utf8x]{inputenc}
\usepackage[a4paper,width=5.5in]{geometry}
\usepackage[bookmarks=true,bookmarksopen=true,bookmarksnumbered=true,
           pdfborder={0 0 0},colorlinks=true,linkcolor=blue]
           {hyperref}
\usepackage{wrapfig}
\usepackage[pdftex]{graphicx} 
\usepackage{romanian}
\usepackage{lastpage}
\usepackage{fancyhdr}
\pagestyle{fancy}

\cfoot{\thepage\ of \pageref{LastPage}}
\lhead{Titlul Lucrarii de licenta}
\rhead{Autorul lucrarii de licenta}
\rightmark
\renewcommand{\sectionmark}[1]{\markright{#1}{}}
\newcommand{\HRule}{\rule{\linewidth}{0.5mm}} 
\begin{document}

\begin{titlepage}

\begin{center}


% Upper part of the page
\includegraphics[width=0.5\textwidth]{Screenshot.png}\\[1cm]    

\textsc{\LARGE Nume student}\\[0.5cm]


\vfill 
% Title
\HRule \\[0.4cm]
{ \huge \bfseries Titlul lucrarii de licenta}\\[0.4cm]

\HRule \\[1.5cm]

% Author and supervisor

\begin{flushleft} \large
\emph{Coordonator} \textsc{Prof. Dr. Viorel Negru}
\end{flushleft}

\begin{flushleft} \large
\emph{Co-Tutel'a} \textsc{Asist. Drd. Ovidiu Aritoni}
\end{flushleft}

\begin{flushleft} \large
\emph{Available at:} \htmladdnormallink {http://web.info.uvt.ro/$\sim$numesite} {http://web.info.uvt.ro/~oaritoni} 
\end{flushleft}
\vfill
% Bottom of the page
{\large Universitatea de Vest din Timi'soara \\ 2011}

\end{center}

\end{titlepage}

\newpage

\textbf{General Overview}

  The program takes as input a file containing timestamps and values
  and outputs 2 files with patterns and their timestamps,
  respectively pattern frequencies and their time intervals.

  Javadoc (API) is in the \verb/help/ folder.

  The main class is \verb/Patters/, which is \verb/public/.

  To compile, type: \verb/javac Patterns.java/

  To run, type: \verb/java Patterns inputFile [patternLength]/

  \verb/inputFile/ and \verb/patternLength/ are command line arguments.
  Only the input file (the first argument) is mandatory, the other one is optional.
  If you specify two command line arguments, the first is the input file
  and the second is the pattern length.
  If you specify no argument or more than two, the usage message is displayed:

  \noindent
  \verb/Usage: java Patterns inputFile [patternLength]/

  \noindent
  and the program exits.

  The pattern length must be an integer greater than or equal to 2.
  This argument is optional, so it has a default value specified in the
  program: 5.
\\
\\
\textbf{Responsible:} Ovidiu Aritoni 
\\
\\
\textbf{Team Members:} Ovidiu Aritoni, Ionel Popescu, Alice Aritoni. 
\\ 
\\
\textbf{Versions}
\\
\\
\begin{tabular}{|c|c|c|c|}
\hline 
Version & Author & Description & Date\tabularnewline
\hline
\hline 
1.0 & Ovidiu Aritoni & The first overview. & 15.10.2010\tabularnewline
\hline
\end{tabular}
\\
\\
\\
\textbf{Approved, }
\newpage 
\tableofcontents
\newpage
\listoffigures
\newpage
\listoftables
\newpage
\section{How the program works}

  The program source code is contained within the file \verb/Patterns.java/
  The name of the file is the same as the name of the public class in it,
  except for the extension.
  
  There are 3 classes defined in this file: \verb/Patt/ describing a pattern,
  \verb/Freq/ for frequencies, and the main class \verb/Patterns/. The first 2 classes
  have \verb/package/ access, while the last is \verb/public/.
  
  \subsection{Patt class}
  
  The \verb/Patt/ class holds data relevant to a pattern (timestamps and values).
  Each pattern has the following fields and methods:
  
  \noindent
  Fields:
  \begin{itemize}
   \item an \verb/id/ (an \verb/integer/), starting with 1.
   \item \verb/patternLength/, the length of the pattern, initialized with a default
    value in the \verb/main()/ method of \verb/Patterns/ or provided as a command line argument
    (optional). The value for the pattern length provided on the command line overrides
    the default value.
   \item timestamps:
    \begin{itemize}
     \item \verb/startTimestamps/, a \verb/List/ of \verb/String/s representing the start
      timestamps for each time the pattern is encountered in the input file;
     \item \verb/endTimestamps/, also a \verb/List/ of \verb/String/s, this time representing
      the timestamps for each last value (out of \verb/patternLength/ values that compose a pattern)
      for that pattern.
    \end{itemize}
    These types of timestamps are in equal number and provide a framework for the appearances of
    the pattern in the input file. The timestamps appear at roughly regular intervals in the input
    file (tens of seconds or minutes). Knowing the start and end times for each appearance of the
    pattern (and eventually the pattern length), we can easily extract the timestamps and trace out
    the evolution for each pattern.
   \item the \verb/values/ that compose the pattern (in \verb/patternLength/ number), which are
    concatenated here to be easier to print them (or write them in a file). It's also easier and
    faster to compare 2 \verb/String/s than \verb/patternLength/ \verb/integer/s. These \verb/values/
    are used to compare patterns to determine when a new pattern appears (see below the \verb/main()/
    method of the public class \hyperlink{patt}{Patterns}).
  \end{itemize}
  
  \noindent
  Methods:
  \begin{itemize}
   \item There is only one constructor, which has arguments (non-default constructor) that
    initialize the private fields \verb/id/ and \verb/patternLength/.
   \item There are getters and setters for all the fields, except for \verb/patternLength/,
    which cannot be modified after it is initialized in the constructor.
   \item There are also 2 methods that add a new start, respectively end timestamp.
   \item The method \verb/toString()/ overrides the corresponding method of the extended class
    \verb/java.lang.Object/. The method returns the fields along with some relevant messages
    for each.
  \end{itemize}

  \subsection{Freq class}
  
  The \verb/Freq/ class represents the frequency of a pattern.
  It has the following fields and methods:
  
  \noindent
  Fields:
  \begin{itemize}
   \item \verb/pattID/, the pattern ID to which the current frequency object is connected.
   \item \verb/freq/, the frequency in absolute value of the pattern (how many times the pattern
    appeared in the input file).
   \item timestamps:
    \begin{itemize}
     \item \verb/startTime/, the start timestamp for the first encounter of the pattern
      in the input file;
     \item \verb/endTime/, the end timestamp for the last encounter of the pattern
      in the input file.
    \end{itemize}
    These 2 timestamps provide a framework for all the appearances of the pattern in the input file.
  \end{itemize}
  
  \noindent
  Methods:
  \begin{itemize}
   \item There is only one constructor, which has an argument (non-default constructor) that
    initializes the private field \verb/pattID/.
   \item There are getters and setters for all the fields.
   \item There is also a method that increments the frequency.
   \item The method \verb/toString()/ overrides the corresponding method of the extended class
    \verb/java.lang.Object/. The method returns the fields along with some relevant messages
    for each.
  \end{itemize}

  \hypertarget{patt}{\subsection{Patterns main class}}
  
  The \verb/public/ class \verb/Patterns/ is the main class of the program.
  Is the only class with a \verb/main()/ method. Besides this, it has one other method,
  \verb/writeFile()/, which is used to generate the 2 output files, for patterns, respectively
  frequencies.
  
  \noindent
  The input file is read one line at a time, which is split in 2 tokens:
  \begin{itemize}
   \item the timestamp (a \verb/String/) and
   \item the value (an \verb/integer/).
  \end{itemize}

  The splitting is done with the method \verb/split()/ of \verb/java.lang.String/, which
  receives a regular expression as an argument. In our case, the line read from the file
  is split around comma, which separates the timestamp and the value.
  
  The timestamps and the values are put in 2 lists: \verb/timestamps/, respectively
  \verb/pattValues/. These will be used to initialize the data in the patterns and
  frequencies objects and to work easier with other objects that contain only indices
  in the above lists.
  
  Patterns and frequencies are organized each in a \verb/TreeMap/ to be kept sorted,
  where the keys are starting timestamps (represented as \verb/integer/s that are
  indices in the \verb/timestamps/ list). The point of this is that the patterns may not
  be discovered in the order they appear in the input file: pattern 3 might be discovered
  after 1, etc. A pattern is found when there is a match of \verb/patternLength/ values
  in the input file: the same \verb/patternLength/ values repeat.
  
  As an example, let's say the input file contains values in the following order:
  \verb/5 7 8 2 4 3 1 2 0 2 4 3 5 7 8/. Let's also assume the pattern length is 3.
  Then, the pattern \verb/5 7 8/ (although the first appearing in the file) is found after
  \verb/2 4 3/. This happens because \verb/2 4 3/ is the first sequence of 3 digits that repeats.
  As such, the repetition of \verb/5 7 8/ is found after it.
  
  The \verb/TreeMap/ holding the patterns (the same applies to frequencies) is kept sorted
  after the keys representing the position of the first starting timestamp. In our example,
  the 2 starting timestamps would have the positions 3, respectively 0 in the \verb/timestamps/
  \verb/List/ (positions start at 0). In the \verb/TreeMap/, they would be kept sorted, such that
  the pattern with ID 1 (the first discovered, \verb/2 4 3/, with starting timestamp on position 3)
  appears after the pattern with ID 2 (\verb/5 7 8/, the second found, with starting timestamp
  on position 0).
  
  At the end, the IDs of the patterns must be changed to reflect the order the patterns appear in
  the input file. In our example, the pattern \verb/5 7 8/, the first in the \verb/TreeMap/ as it
  has the starting timestamp on position 0, would change the ID from 2 to 1, and the pattern
  \verb/2 4 3/ would correspondingly get the ID 2.
  
  This is done in the \verb/main()/ method of \verb/Patterns/ just before calling the method
  \verb/writeFile()/ at the end of it.
  
  The patterns are found by matching the last value of the pattern. The condition for finding
  patterns is that the last value of the pattern must be at least on position \verb/patternLength-1/.
  A counter is used to keep track of the lines read from the input file.
  The counter is initialized with 0 and incremented after each line is read and processed.
  As such, after reading \verb/patternLength/ lines, it has the value \verb/patternLength/.
  
  To have at least a repeating pattern, the counter must be at least 2 times the pattern length.
  Besides, between the repeating values must be at least the distance \verb/patternLength/.
  
  Considering the previous conditions are fulfilled, all equal values are compared.
  The \verb/values/ \verb/TreeMap/ holds the values and their positions.
  The values to be compared are taken from the \verb/List/ \verb/pattValues/.
  If the values are not equal, they are just added to the \verb/TreeMap/ \verb/values/ and
  the counter is incremented.
  In our example, the first repeating value is 2, found on positions 3 and 7.
  
  Then the previous values are compared in the limit \verb/patternLength/. In our case,
  the previous values are 8 and 1, found on positions 2, respectively 6. These are not equal,
  so we move on, as we didn't find a pattern. 
 
  Next, we compare 2 on position 9 with 2 on position 3 (between 2 on position 9 and 2 on
  position 7 there is not enough room for a pattern: $9-7=2<3$, the pattern length).
  Again, no match, as the previous values are different (0 on position 8, compared with 8 on
  position 2).
  
  The next repeating value is 4, on positions 10 and 4. The previous values are both 2 (on
  positions 9 and 3), but the pattern length is 3, so we need one more value to find a pattern.
  The previous values are 0 and 8, on positions 8, respectively 2, so again no match.
  
  We stop at the next repeating value, 3, where we have a match: 3 on positions 11 and 5,
  4 on positions 10 and 4, and 2 on positions 9 and 3. The found pattern is \verb/2 4 3/.
  
  These values are stored in an array of \verb/integer/s initialized from the end to the beginning
  as the values are found: 3, 4, 2. The array is used to build a \verb/String/ using a
  \verb/StringBuilder/. These are used for efficiency. The result is turned into a \verb/String/
  by calling the method \verb/toString()/ of the \verb/StringBuilder/ object (inherited from
  \verb/java.lang.Object/).
  
  After a pattern is found, it is compared with existing patterns, stored in the \verb/TreeMap/
  \verb/pattMap/. In order to compare the patterns, these are extracted from the map by calling
  the method \verb/values()/, which returns a \verb/Collection/ of the values held in the map,
  in our case patterns (\verb/Patt/ objects). This is \verb/pattList/.
  
  The found pattern is compared with those in the list by comparing the values each holds.
  These values are stored concatenated as a \verb/String/ in the \verb/values/ field, accessible
  by calling the corresponding getter method.
  
  If a match is found, the frequency of that pattern is incremented (provided the start timestamp
  of the last time the pattern is met is newer than the end timestamp of the previous time the
  pattern was encountered) -- in the corresponding \verb/Freq/ object, that with the same ID as
  the pattern (\verb/Patt/) object. The frequency object is updated with the last end timestamp,
  and the new start and end timestamps are added to the pattern object as well.
  
  If the pattern is new, it is added to the map, along with the corresponding timestamps
  and a frequency object is added as well to the frequency map \verb/freqMap/.
  First, the pattern has frequency 2, which is also the condition to discover a pattern:
  to repeat once.

 % \section{Output patterns}

 

 % \section{Output pattern frequencies}

 

\end{document}
